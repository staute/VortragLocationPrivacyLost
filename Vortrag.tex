\documentclass[xcolor=dvipsnames]{beamer}
\usetheme{Berkeley}
\usecolortheme[named=ForestGreen]{structure}
\setbeamercovered{transparent}
\beamertemplatenavigationsymbolsempty

\usepackage[ngerman]{babel}
\usepackage[utf8]{inputenc}
\title{Vortrag „Loss of Location Privacy“}
\author{Stephan Latta \& Stefan Taute}
%\institute{\hspace{1cm}\includegraphics[width=80pt]{images/fhs.pdf} \hspace{1cm} \includegraphics[scale=0.4]{images/planetic2.pdf}} 
\institute{\includegraphics[width=80pt]{images/fhs.pdf}}
\date{11. Dezember 2012}






\begin{document}

%\begin{frame}{Bild Beispiel}
%\begin{center}
%\includegraphics[scale=0.3]{images/fhs.pdf}
%\end{center}
%\end{frame}

%\begin{itemize}
%\item \href{run:iphone.mp4}{Video: Iphone Tracking Discussion}
%\end{itemize}
%\movie[label=cells,width=\textwidth,height=.8\textheight,poster,showcontrols]{}{iphone.mp4}
%\movie[poster,controls=false]{}{iphone.mp4}

\begin{frame}[plain]
 \titlepage
\end{frame}

\begin{frame}{Thema}
\begin{center}
  \huge \textbf{„Location Privacy“\\ \vspace{.5cm} Verlust der Privatsphäre im Mobilfunkzeitalter}
\end{center}
\end{frame}

\begin{frame}{Gliederung}
  \tableofcontents[pausesections,hideallsubsections]
\end{frame}

\section{Einleitung}
\begin{frame}{Einleitung}
\begin{itemize}
  \item Hintergrund
  \item Zielsetzung
\end{itemize}
\end{frame}

\subsection{Hintergrund}
\begin{frame}{Hintergrund}
\begin{itemize}
  \item "Location Privacy" häufiger Begriff in Datenschutzdebatten
  \item mögliche Risiken, Gefahren des Verlusts der örtlichen Privatsphäre
  \item heutzutage besitzen die meisten Menschen ein Smartphone, somit jederzeit lokalisierbar
  \item entscheidene Unternehmen sind Apple und Google
  \item Brisanter Fund April 2011: Alasdair Allan und Peter Warden finden Datei namens "consolidated.db"
\end{itemize}
\end{frame}

\begin{frame}{Hintergrund}
\begin{itemize}
  \item Apple zeichnet außerdem MAC-Adressen und Signalstärken von Access Points, Speicherung in Kombination mit weiteren Geo- und Zeit-Daten
  \item Apple zeichnet desweiteren IDs und Signalstärken von Mobilfunkmasten auf, Speicherung in Kombination mit IPhone Geo-Daten
  \item Apple versicherte die Daten werden nur anonymisiert genutzt, die Daten werden nur zur Verbesserung von Location Based Services verwendet
\end{itemize}
$\Rightarrow$ \textbf{Je besser die Lokalisierungstechniken, desto schwieriger die Anonymität und Privatsphäre zu wahren.}
\end{frame}

\begin{frame}{Hintergrund}
\begin{center}
  \huge \textbf{Video:\\\vspace{1cm} ``iPhone Tracking Discussion''}
\end{center}
\vspace{2cm}
\begin{center}
  \href{run:iphone.mp4}{\beamerbutton{Start Video}}
\end{center}
\end{frame}

\subsection{Zielsetzung}
\begin{frame}{Zielsetzung}
\begin{itemize}
  \item Einblick in das Thema "Location Privacy"
  \item Grundlage war der wissenschaftliche Artikel "The Loss of Location Privacy in the Cellular Age" (Stephen B. Wicker, August 2012 CACM)
  \item Grundlagen zum Verstehen der Thematik
  \item Gefahren \& Risiken
  \item Möglichkeiten zum Schutz durch direkte Anonymisierung
\end{itemize}
\end{frame}

\section{Grundlagen}
\begin{frame}{Grundlagen zur Thematik}
\begin{itemize}
  \item Ortungstechnologie
  \begin{itemize}
    \item Mobilfunküberwachung \& E911
    \item GPS
    \item Alternative zu GPS
  \end{itemize}
  \item Location Based Services
  \item Location Based Advertising
  \item Privatsphäre \& Anonymisierung
  \item Ort
\end{itemize}
\end{frame}

\subsection{Ortungstechnologie}
\subsubsection{Mobilfunküberwachung \& E911}
\begin{frame}{Mobilfunküberwachung \& E911}
\begin{itemize}
  \item Mobilfunknetze grundsätzlich darauf ausgelegt Endgeräte zu tracken, ursprünglich nur um nächstgelegenen Mast zu finden
  \item 1996 erste Versuche die Ortung zu verfeinern (E911)
  \item E911 sollte die Mobilfunkanbieter zwingen die Ortsinformation bei einem 911-Anruf an die Notrufzentrale zu übermitteln
  \item E911 war sozusagen der Grundstein für die Ortung im Mobilfunksektor
\end{itemize}
\end{frame}

\subsubsection{GPS}
\begin{frame}{GPS}
\begin{itemize}
  \item heutzutage verfügen Smartphones über GPS womit die Ortung noch exakter ist
  \item das \textbf{G}lobal \textbf{P}ositioning \textbf{S}ystem ist ein Satelliten gestütztes Ortungssystem
  \item 
\end{itemize}
\end{frame}

\subsubsection{Alternative zu GPS}
\begin{frame}{Alternative zu GPS}
\begin{itemize}
  \item \dots
\end{itemize}
\end{frame}

\subsection{Location Based Services}
\begin{frame}{Location Based Services}
\begin{itemize}
  \item \dots
\end{itemize}
\end{frame}

\subsection{Location Based Advertising}
\begin{frame}{Location Based Advertising}
\begin{itemize}
  \item \dots
\end{itemize}
\end{frame}

\subsection{Privatsphäre \& Anonymisierung}
\begin{frame}{Privatsphäre \& Anonymisierung}
\begin{itemize}
  \item \dots
\end{itemize}
\end{frame}

\subsection{Ort}
\begin{frame}{Ort}
\begin{itemize}
  \item \dots
\end{itemize}
\end{frame}

\section{Location Privacy}
\begin{frame}{Location Privacy - Bedenken, Risiken \& Gefahren}
\begin{itemize}
  \item \dots
\end{itemize}
\end{frame}

\section{Location Anonymity}
\begin{frame}{Location Anonymity}
\begin{itemize}
  \item Shannon-theoretischer Ansatz eines Models
  \item Anonymisierte Location Based Services
\end{itemize}
\end{frame}

\subsection{Ansatz Shannon}
\begin{frame}{Shannon-theoretischer Ansatz eines Models}
\begin{itemize}
  \item \dots
\end{itemize}
\end{frame}

\subsection{Anonymisierte LBS}
\begin{frame}{Anonymisierte Location Based Services}
\begin{itemize}
  \item \dots
\end{itemize}
\end{frame}

\section{\dots}
\begin{frame}{\dots}
\begin{itemize}
  \item \dots
\end{itemize}
\end{frame}

\section{Fazit}
\begin{frame}{Fazit}
\begin{itemize}
  \item \dots
\end{itemize}
\end{frame} 

\section{Fragen}
\begin{frame}{Fragen}
\begin{minipage}{\textwidth}
  \begin{center}
    \Huge \textbf{Vielen Dank für ihre Aufmerksamkeit!}
    \vspace{1cm}
  \end{center}
\end{minipage}
\begin{minipage}{\textwidth}
  \begin{center}
    \transduration<+->{0.75}
    \only<+>{\Huge \textbf{Fragen?}}
  \end{center}
\end{minipage}
\end{frame}

\end{document}
